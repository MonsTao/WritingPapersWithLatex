\documentclass{beamer}
\usepackage{ctex,tcolorbox,color,xcolor}
\usepackage[ruled]{algorithm2e}
\usepackage{hyperref}
\hypersetup{hidelinks,
	colorlinks=true,
    linkcolor=blue,
	urlcolor=cyan,
	pdfstartview=Fit,
	breaklinks=true}

\usetheme{Malmoe}
\usecolortheme{dolphin}
\usefonttheme{serif}

\title{Writing Papers With \LaTeX{}}
\subtitle{基于\LaTeX{}的论文写作快速入门}
\author{Yao Zhengji}
\date{\today}
\institute{NEU}


\begin{document}

\frame{\maketitle}

\begin{frame} 
    \frametitle{\LaTeX{}简介} 
    \begin{itemize} 
        \item \LaTeX{}是一款基于\TeX{}的专业排版软件
        \item 相比于Word这样“所见即所得”的文字处理软件来说,\LaTeX{}属于编程实现。即tex文件是源代码,pdf文件是编译结果。
        \item 为什么学习/使用\LaTeX{}?
        \begin{itemize}
            \item \LaTeX{}可以便捷、清晰、美观地展示出公式、编号、引用等内容
            \item 大量科研会议、期刊投稿要求\LaTeX{}文件
        \end{itemize}
    \end{itemize} 
\end{frame}

\begin{frame}
    \frametitle{目录}
    \tableofcontents
\end{frame}

\begin{frame} 
    \frametitle{快速开始:Overleaf} 
    \begin{itemize} 
        \item \href{https://cn.overleaf.com/}{Overleaf官网}(支持中文)
        \item 在线的\LaTeX{}编译器
        \item 优点:无需配置环境;可跨设备查看、编译;
        \item 缺点:必须在有良好网络的环境使用;编译速度较慢;可能出现难以调试的bug
    \end{itemize} 
\end{frame}

\begin{frame} 
    \frametitle{本地的环境配置} 
    \begin{itemize} 
        \item 本地\LaTeX{}编译环境
        \begin{itemize}
            \item 进入\href{https://www.latex-project.org/}{\LaTeX{}官网},在其中的Get页面可以根据系统不同下载不同的环境安装包。一般安装包为iso文件,如果是Win10以上系统,直接运行即可
            \item 推荐使用TeXLive,其扩展较全,避免后续再下载扩展包的麻烦,如果流量告急,可以先用MikTex
            \item TeXLive文件约4G,如下载较慢,可以在\href{https://mirrors.tuna.tsinghua.edu.cn/CTAN/systems/texlive/Images/}{镜像站}下载
        \end{itemize}  
        \item 本地\LaTeX{}编辑器
        \begin{itemize}
            \item 推荐使用\href{https://code.visualstudio.com/}{VSCode},功能非常丰富,且可用于处理诸多其他文件类型
        \end{itemize}
        \item 优点:本地编译;编辑器可自定义程度较高
        \item 缺点:配置过程较为复杂(配置过程可参见下一页)
    \end{itemize} 
\end{frame}

\begin{frame} 
    \frametitle{VSCode的配置} 
    \begin{itemize} 
        \item 在VSCode的扩展页面中安装LaTeX Workshop插件,在打开tex文件后,可以直接编译(快捷键为Ctrl+Alt+B)
        \item 如果编译不通过,可能是编译设置问题,可以直接在“Build LaTeX project”菜单中选择{xelatex$\rightarrow$biber$\rightarrow$xelatex*2}
        \item 当前版本的VSCode可以直接左右分标签页来分别浏览tex文件与pdf文件,故不必安装SumatraPDF
        \item VSCode推荐设置:
        \begin{itemize}
            \item 设置中可以勾选Editor: Word Wrap,开启自动换行
            \item 设置-主题中可以更改界面配色与关键词高亮配色(需要主题支持\LaTeX{}),笔者使用的是Noctis主题的Hibernus配色
            \item 设置中可以找到latex-workshop.latex.autoBuild.run,其中可以设置是否在保存时编译(可以每次Ctrl+S保存时自动更新pdf内容)
        \end{itemize}
    \end{itemize} 
\end{frame}


\begin{frame} 
    \frametitle{文档模板的应用} 
    \begin{itemize} 
        \item 不建议从空文档开始论文写作,尽量从模板开始
        \item Overleaf的Template页面有大量免费模板,可以在初开始写作时应用
        \begin{itemize}
            \item Overleaf的模板可以在打开后,从左上角的菜单内选择下载源码到本地,即可实现线上转本地编辑
        \end{itemize}
        \item 在确定投稿目标后,即可在对应的期刊/会议的网站上下载到其所要求的模板
    \end{itemize} 
\end{frame}

\begin{frame} 
    \frametitle{标题与作者} 
    \begin{itemize} 
        \item 在已经应用模板的情况下,修改$\backslash$author中的作者名和$\backslash$authormark中的作者信息即可
        \item $\backslash$author中作者名后的大括号内填写数字代表第几作者,填写星号代表通信作者
    \end{itemize} 
\end{frame}

\begin{frame} 
    \frametitle{公式} 
    \begin{itemize} 
        \item \LaTeX{}中常用的公式有两种形式:行内公式与行间公式
        \item 行内公式在两侧各加一个\$即可,行间公式需要两侧各加两个\$
        \item 两侧各加两个\$的方法可能导致编译问题,使用行间公式时建议左侧加$\backslash$[右侧加$\backslash$]
        \item 在公式结尾加$\backslash$tag\{XXX\}可以给公式编号
        \item \[a=b+c \tag*{5.1}\]
        \item 如果想在文字中间插入符号,最好左右各添\$符号以确保显示正常
        \item 如果需要可视化地编辑公式,可使用Word编辑公式后复制公式粘贴进\LaTeX编辑页面,一般会自动粘贴为公式代码(不要忘记加符号)。或使用\href{https://latex.codecogs.com/eqneditor/editor.php?lang=zh-cn}{在线公式编辑器}
    \end{itemize} 
\end{frame}

\begin{frame} 
    \frametitle{图片} 
    \begin{itemize} 
        \item 插入一个图片的命令一般以{$\backslash$begin\{figure\}[!h]},以$\backslash$end\{figure\}结束
        \item 常用$\backslash$centering居中图片,用$\backslash$label增加图片标签
        \item 在使用模板的情况下,一般只需要输入图片文件名
        \item 所有插入图片应在同一文件夹中,如果图片和tex文件不在同一路径下,须在开头用graphicspath命令指定路径
    \end{itemize} 
\end{frame}

\begin{frame} 
    \frametitle{表格} 
    \begin{itemize} 
        \item 不建议使用\LaTeX{}命令制表
        \item 直接使用Excel制表,之后使用\href{https://tableconvert.com/}{TableConvert}进行转化即可
        \item TableConvert的页面上也可继续编辑
    \end{itemize} 
\end{frame}

\begin{frame} 
    \frametitle{引用} 
    \begin{itemize} 
        \item 可以在文档中直接书写引用,或使用BibTex
        \item 如果直接书写,以begin\{thebibliography\}\{99\}开头,以end\{thebibliography\}结尾即可。中间插入bibitem\{XXX\}YYY,其中XXX为引用时的标签,YYY为文章与作者信息
        \item 如果使用BibTex,在参考文献处直接写bibliographystyle{plain}和bibliography{ref}即可
        \item 在文中引用时,输入cite\{XXX\}
        \item 一般不需要对着论文名称和作者名称手敲引用,将想要引用的文章名在谷歌学术搜索,只要是能搜到的文章都可一键引用
    \end{itemize} 
\end{frame}

\begin{frame} 
    \frametitle{其他参考} 
    \begin{itemize} 
        \item \href{https://zhuanlan.zhihu.com/p/38178015}{本地安装与配置过程}
        \item \href{https://zhuanlan.zhihu.com/p/67182742}{使用Overleaf}
        \item \href{https://blog.csdn.net/NSJim/article/details/109045914}{公式}
        \item \href{https://blog.csdn.net/qq_31347869/article/details/103832190}{图片}
        \item \href{https://blog.csdn.net/juechenyi/article/details/77116011}{表格}
        \item \href{https://ctan.org/tex-archive/info/lshort/chinese}{较全面的\LaTeX{}指南}(建议下载保存)
        \item \href{https://github.com/MonsTao/WritingPapersWithLatex}{本演示的源码}
    \end{itemize} 
\end{frame}


\end{document}